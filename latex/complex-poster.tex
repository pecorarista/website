\documentclass[unicode,colorlinks]{beamer}
\usepackage[orientation=landscape,size=a0,scale=1.4]{beamerposter}
\usetheme[footertext={フッターに何か適当なことを書くこともできます.}]{SimplePoster}
\usepackage{lipsum}
\usepackage{amsmath,amssymb}
\newcommand{\absolute}[1]{\left|#1\right|}
\usepackage{tikz}
\usepackage{pgfplots}
\usetikzlibrary{arrows.meta}
\usepackage{tabularx}
\usepackage{listings}
\usepackage[no-math]{luatexja-fontspec}
\usepackage{unicode-math}
\ltjsetparameter{%
    jacharrange={%
        -2, % Exclude Greek and Cyrillic letters.
        -3  % Punctuations and Miscellaneous symbols.
    }
}
\usepackage{babel}
\babelprovide[main]{japanese}
\setsansfont[BoldFont={FrutigerLTStd-Bold}]{OpenSans}
\renewcommand{\familydefault}{\sfdefault}
\renewcommand{\kanjifamilydefault}{\gtdefault}
\newfontfamily\ipafont{FreeSerif}
\babelprovide{french}
\babelfont[french]{sf}[Script=Cyrillic]{OpenSans}
\setsansjfont[BoldFont={HiraKakuPro-W6}]
{HiraKakuPro-W3}
\setmonofont{Source Code Pro}
\makeatletter
\mode<presentation>{\beamer@suppressreplacementstrue}
\makeatother
\usefonttheme{professionalfonts}
\setmathfontface\asanamath{Asana Math}
\setoperatorfont\asanamath
\usepackage{arabluatex}
\newfontfamily\arabicfont[%
    Script=Arabic,     % enable ligatures
    RawFeature={%
        +anum,         % use eastern arabic numerals
        +ss05}         % put kasrah below shadda
]{Amiri}
\SetTranslitFont{\ipafont}
\SetTranslitStyle{\upshape}
\SetTranslitConvention{dmg}
\usepackage{booktabs}
\setlength\heavyrulewidth{3pt}
\newcolumntype{C}{>{\centering\arraybackslash}X}
\lstset{%
    language={Python},
    basicstyle={\ttfamily},%
    keywordstyle={\color{red}\ttfamily\bfseries},%
    commentstyle={\color{gray}\ttfamily},
    stringstyle={\ttfamily},
    xleftmargin=2em,
}
\pgfplotsset{%
    compat=newest,
    xlabel near ticks,
    ylabel near ticks
}
\title{ポスターのサンプル}
\author{著者名}
\institute{所属機関名}
\date{\today}
\begin{document}
\begin{frame}[fragile]
\begin{columns}[t]
    \begin{column}{.32\linewidth}
        \begin{block}{はじめに}
            \LaTeX{} (Beamer)でポスターを作りましょう.
        \end{block}
        \begin{block}{数式}
            \LaTeX{}で作るので数式の挿入が簡単にできます.

            (最近はPowerPointでもできるらしいですが...)

            \bigskip
            \bigskip

            \textbf{Lebesgueの収束定理} (小谷, 2005)

            $(f_n)_n$は$X$上可測関数列で,ある非負可積分関数$g$により
            \begin{align*}
                \absolute{f_n(x)} \leq g(x) \quad (x \in X)
            \end{align*}
            が成り立つとする.極限 $\lim_{n \to \infty}f_n(x) = f(x)$ が存在するなら
            \begin{align*}
                \int_X \lim_{n \to \infty}f(x) \mu(dx)
                = \lim_{n \to \infty}\int_X f_n (x) \mu(dx)
            \end{align*}
        \end{block}
        \begin{block}{ダミーテキスト}
            \lipsum[1]
        \end{block}
        必ず\texttt{beamercolorbox}の中に書かなければならない,というわけではありません.
    \end{column}
    \begin{column}{.32\linewidth}
        \begin{block}{グラフ}
            \verb+\includegraphics+でRなどで作成したグラフを取り込むことはもちろん,
            TikZでLaTeXソースコード中にグラフのコードを書くこともできます.

            \bigskip

            \tikzset{%
                declare function={%
                    normcdf(\x,\m,\s)=1/(1 + exp(-0.07056*((\x-\m)/\s)^3 - 1.5976*(\x-\m)/\s));
                },
                % http://tex.stackexchange.com/questions/5461/is-it-possible-to-change-the-size-of-an-arrowhead-in-tikz-pgf
                % TikZ manual p. 202
                >={Straight Barb[width=5mm,length=5mm]}
            }
            \begin{center}
                \begin{tikzpicture}
                    \begin{axis}[
                        samples=500,
                        domain=-10:10,
                        width=30cm, height=20cm,
                        xmin=-10.5, xmax=10.5,
                        ymin=-0.3, ymax=1.5,
                        axis x line=center,
                        axis y line=center,
                        xlabel={$x$},
                        ylabel={$y$},
                        axis line style={->, ultra thick},
                        xtick={-5,0,5},
                        ytick={0,0.5}]
                    \addplot[color=red,ultra thick] plot{normcdf(0.626657*x,0,1)} node[pos=0.4](cum){};
                    \addplot[color=blue,ultra thick,dashed] plot (\x,{1/(1+exp((-1)*x))}) node[pos=0.6](a){} ;
                    \node [color=blue,below right] at (a) {${\displaystyle y=\frac{1}{1+\exp(-x)}}$};
                    \node [color=red,above left] at (cum) {${\displaystyle y=\varPhi\left(\sqrt{\frac{\pi}{8}}x\right)}$};
                    \addplot[dashed,ultra thick] plot (\x,{1}) node[pos=0.5](b){};
                    \node [above left] at (b) {$y=1$};
                    \node at (0,0) [anchor=north west]{$O$};
                    \end{axis}
                \end{tikzpicture}
            \end{center}
        \end{block}

        \begin{block}{プログラム}
            \texttt{listings}パッケージを読み込んでプログラムのソースコードを書くことができます.
            フレームに\texttt{fragile}オプションを渡す必要があるので気をつけましょう.

            \bigskip

            \begin{lstlisting}
# This function does not always work!
def plural(word):
    if word.endswith('y'):
        return word[:-1] + 'ies'
    elif word[-1] in 'sx':
        return word + 'es'
    else:
        return word + 's'
            \end{lstlisting}
        \end{block}
    \end{column}
    \begin{column}{.32\linewidth}
        \begin{block}{文字}
            アラビア文字など日本語の文書中に混在させにくい文字も
            \LaTeX{}のパッケージを利用して簡単に入力することができます。

            \bigskip
            \bigskip


            \begin{ipafont}
                kaanat rijħu ʃʃamaali tatadʒaadalu wa ʃʃamsa fij ʔajjin minhumaa kaanat ʔaqwaa min alʔuxraa
            \end{ipafont}

            \bigskip
            \bigskip

            \begin{arab}[fullvoc]
                kAnat rI.hu 'l-^samAli tatajAdalu wa 'l-^samsa fI 'ayyiN minhumA kAnat 'aqw_A mina 'l-'uxr_A.
            \end{arab}

            \bigskip
            \bigskip

            \hfill {\small 『国際音声記号ガイドブック』「アラビア語」より}

            \bigskip
            \bigskip

            【アラビア文字部分の実際の入力】

            \verb+kAnat rI.hu 'l-^samAli tatajAdalu+
            \verb+wa 'l-^samsa fI 'ayyiN minhumA kAnat+
            \verb+'aqw_A mina 'l-'uxr_A.+
        \end{block}

        \begin{block}{実験結果}
            過去に作った\LaTeX ファイルがそのまま流用できます.

            \bigskip
            \bigskip

            \begin{center}
                \begin{tabularx}{0.5\textwidth}{CC}
                    \toprule
                             & スコア \\
                    \midrule
                    手法1    & 0.11   \\
                    手法2    & 0.22   \\
                    手法3    & 0.33   \\
                    手法2    & 0.44   \\
                    \bottomrule
                \end{tabularx}
            \end{center}
        \end{block}


        \begin{block}{参考文献}
            \begin{enumerate}
                \item 小谷眞一『測度と確率』, 岩波書店 (2005).
                \item 国際音声学会編, 竹林滋・神山孝夫訳『国際音声記号ガイドブッ

                    ク』, 大修館書店 (2003).
            \end{enumerate}
        \end{block}

        \href{https://github.com/deselaers/latex-beamerposter}{\texttt{beamerposter}パッケージ}
        の\texttt{examples}を参考にして自分用の\texttt{.sty}ファイルを作成しましょう.
    \end{column}
\end{columns}
\end{frame}
\end{document}
