\documentclass[12pt,a4paper]{ltjsarticle}
\usepackage[no-math]{luatexja-fontspec}
\usepackage{babel}
\babelprovide{russian}
% \babelprovide{polutonikogreek}
\ltjsetparameter{%
    jacharrange={%
        -2, % Exclude Greek and Cyrillic letters.
        -3  % Punctuations and Miscellaneous symbols.
    },
    alxspmode={`/,allow}
}
\babelfont[russian]{sf}[Script=Cyrillic]{OpenSans}
\usepackage{arabluatex}
\setmainfont[Ligatures=TeX]{Free Serif}
\setmainjfont[BoldFont={RyuminPro-Regular}]{RyuminPro-Light}
\newjfontfamily\chinesefont{Hiragino Sans GB}
\newfontfamily\phonetic{Doulos SIL}
\newfontfamily\translitfont{Free Serif}
\newfontfamily\arabicfont[%
    Script=Arabic,     % enable ligatures
    RawFeature={%
        +anum,         % use eastern arabic numerals
        +ss05}        % put kasrah below shadda
]{Scheherazade-Bold}
\SetTranslitFont{\translitfont}
\SetTranslitStyle{\upshape}  % \upshape, \itshape
\SetTranslitConvention{dmg}  % dmg, loc, arabica
\newcommand\tightlist{\relax}
\newcommand\ArabicWord[1]{\mbox{\arb[novoc]{#1}} \arb[trans]{#1}}
\newcommand\ChineseWord[1]{\begin{chinesefont}#1\end{chinesefont}}
\newcommand\RussianWord[1]{\foreignlanguage{russian}{\textsf{#1}}}
\pagestyle{empty}
\begin{document}
  \begin{itemize}
  \tightlist
  \item
    \ChineseWord{狐狸} húli {\phonetic /xʷuu˥lʲii/} 狐
  \item
    \ChineseWord{春节} chūnjié {\phonetic /ʈʂʰʷən˥tɕee˧˥/} 春節
  \item
    \ChineseWord{近日} jìnrì {\phonetic /tɕin˥˩ʐʐʐ˥˩/} このごろ
  \item
    \RussianWord{убивать} {\phonetic /ʊbʲɪˈvatʲ/} \textsc{impf}
    \textsc{{[}+ acc{]}} 殺す
  \item
    \RussianWord{воздух} {\phonetic /ˈvozdʊx/} \textsc{m} 水
  \item
    \ArabicWord{rI.huN} \textsc{f} 風
  \item
    \ArabicWord{^samAluN} \textsc{m} 北
  
    \begin{itemize}
    \tightlist
    \item
      \ArabicWord{rI.hu 'l-^samAli} 北風
    \end{itemize}
  \item
    \ArabicWord{tatajAdalu} (\arb[novoc]{fI} + etw.) bestreiten
    \textsc{3sg f impf} ← \ArabicWord{tajadala}
  \end{itemize}
\end{document}
